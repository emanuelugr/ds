\documentclass{article}
%Para imagenes
\usepackage{graphicx}
\usepackage{float}
\usepackage{enumitem} % Paquete para personalizar listas
\usepackage[margin=2cm]{geometry} % Ajusta todos los márgenes a 2 centímetros
\usepackage{indentfirst} % Este paquete fuerza la sangría del párrafo en la primera línea
\setlength{\parindent}{1cm} % Ajusta la sangría de los párrafos a 1 cm


\begin{document}
    \begin{titlepage}
        \centering
        {\bfseries\LARGE Universidad de Granada\par}
        \vspace{1cm}
        {\scshape\Large Facultad de Ingeniería Informática \par}
        \vspace{2cm}
        {\scshape\Huge Practica 3: Pruebas software \par}
        \begin{figure}[h]
                \centering
                \includegraphics[width=0.6\textwidth]{logo_UGR.jpg}
                \label{fig:portada}
            \end{figure}
        {\itshape\Large DS: Grupo 1.7\par}
        \vfill
            {\Large  Emanuel Giraldo Herrera\par}
            {\Large  Thomas Lang \par}
            {\Large  Timur Sorokin \par}
            {\Large  Alejando Iborra Morán \par}
        \vfill
        {\Large (2023-2024) \par}
    \end{titlepage}

\tableofcontents

\newpage
\section{Planification of Software tests}
\subsection{Race Selection}

\textbf{Test:}
\begin{itemize}
    \item Select each valid race (Elfo, Enano, Humano, Orco) from the available options.
\end{itemize}

\textbf{Explanation:}\\
This test verifies that the application properly recognizes and processes user selections for character race. It ensures characters can be created with intended racial attributes.

\subsection{Class Selection}

\textbf{Test:}
\begin{itemize}
    \item Select each valid class (Caballero, Ladrón, Mago, Ranger) from the available options.
\end{itemize}

\textbf{Explanation:}\\
Similar to race selection, this test ensures the application handles class selection correctly. It guarantees characters can be created with the functionalities associated with their chosen class. 

\subsection{Character Name}

\textbf{Test:}
\begin{itemize}
    \item Enter a valid character name.
    \item Leave the character name field empty.
    \item Enter a character name with a really long name
\end{itemize}

\textbf{Explanation:}\\
This test verifies the application allows valid character naming and handles empty or excessively long names appropriately. Testing empty and long names helps identify potential issues like crashes or errors when a user doesn't provide a name or exceeds the limit.

\subsection{Character Creation}

\textbf{Test:}
\begin{itemize}
    \item Create a character with valid selections for race, class, and name.
    \item Attempt to create a character with missing selections .
\end{itemize}

\textbf{Explanation:}\\
This test confirms the core functionality of the application: creating characters based on user selections. It ensures characters are created as intended. Testing with missing selections helps identify potential bugs or incomplete functionalities if required fields are left empty.

\subsection{Character List Filtering}

\textbf{Test:}
\begin{itemize}
    \item Filter characters by each available race.
    \item Filter characters by each available class.
    \item Filter characters by both race and class.
\end{itemize}

\textbf{Explanation:}\\
This test verifies the functionality of the character list filtering system. It ensures users can effectively search and find characters based on specific criteria.

\subsection{Character List Sorting}

\textbf{Test:}
\begin{itemize}
    \item Sort characters by name.
    \item Sort characters by race.
    \item Sort characters by class.
\end{itemize}

\textbf{Explanation:}\\
This test verifies the functionality of the character list sorting system. It ensures users can organize their characters based on different criteria for easier browsing.

    \subsection{Character Details Screen}
        \textbf {Test:}
        \begin{itemize}
            \item Create a character and navigate to its details screen.
            \item Verify that the displayed details (name, race, class, attributes) match the created character's information.
        \end{itemize}
        
        \textbf {Explanation:}\\
         This test confirms that the application correctly retrieves and displays information for a selected character. It ensures users can access and review the details of their creations.

    
    \subsection{Character Deletion} 
    
        \textbf{Test:} 
        \begin{itemize}
            \item Create a character.
            \item Delete the created character from the list
            \item Verify that the character is removed from the list and cannot be accessed anymore.
        \end{itemize}
        
        \textbf{Explanation:}\\ 
        This test verifies the functionality of character deletion. It ensures users can manage their character collection and remove unwanted characters.It's also important to check for proper handling during deletion, such as confirmation prompts to avoid accidental character loss.
        
\newpage
\section{Test Case Analysis}
\begin{table}[htbp]
\begin{tabular}{|p{5cm}|p{6cm}|p{6cm}|}
\hline
\textbf{Name of the Test} & \textbf{Conditions} & \textbf{Needed Data} \\ \hline
Valid Race Selection & User selects each available race & - List of available races \\ \hline
Valid Class Selection & User selects each valid class from the available options & - List of available classes \\ \hline
Valid Character Name & User enters a valid character name & - \\ \hline
Empty Character Name & User leaves the character name field empty & - \\ \hline
Long Character Name & User enters long character name & - Character name length limit (if any) \\ \hline
Valid Character Creation & User creates a character with valid selections for race, class, and name & - List of available races \& classes \\ \hline
Missing Selection Creation & User attempts to create a character with missing selections & - List of available races \& classes \\ \hline
Race Filter & User filters characters by each available race & - List of available races \\ \hline
Class Filter & User filters characters by each available class & - List of available classes \\ \hline
Race \& Class Filter & User filters characters by both race and class & - List of available races \& classes \\ \hline
Name Sorting (Ascending) & User sorts characters by name & - A set of characters with different names \\ \hline
Race Sorting (Ascending) & User sorts characters by race & - A set of characters with different races \\ \hline
Class Sorting (Ascending) & User sorts characters by class & - A set of characters with different classes \\ \hline
Character Details View & User creates a character and views its details screen & - \\ \hline
\end{tabular}
\end{table}

\newpage
\section{Test Design}

\begin{table}[htbp]
\small % Reducir el tamaño de la fuente
\begin{tabular}{|p{6cm}|p{2.5cm}|p{3cm}|p{4.5cm}|}
\hline
\textbf{Sorted Test Cases} & \textbf{Test Environment} & \textbf{Test Data} & \textbf{Relation to Test Conditions} \\ \hline
Valid Selections: \newline 1. Select each valid race. \newline 2. Select each valid class. \newline 3. Enter a valid character name. & Mobile Device Simulator / Linux / Windows & - List of available races \& classes & Valid Race Selection, Valid Class Selection, Valid Character Name \\ \hline
Empty/Long Character Name: \newline 1. Leave character name field empty. \newline 2. Enter a long name. & Mobile Device Simulator / Linux / Windows & & Empty Character Name, Long Character Name \\ \hline
Missing Selection Creation: \newline 1. Attempt to create a character with missing race or class. & Mobile Device Simulator / Linux / Windows & - List of available races \& classes & Missing Selection Creation \\ \hline
Character Creation \& Details: \newline 1. Create a character with valid selections. \newline 2. View details of a created character. & Mobile Device Simulator / Linux / Windows & - List of available races \& classes & Valid Character Creation, Character Details View \\ \hline
Filtering \& Sorting: \newline 1. Filter characters by race (individually). \newline 2. Filter characters by class (individually). \newline 3. Filter characters by race \& class combined. \newline 4. Sort characters by name. \newline 5. Sort characters by race. \newline 6. Sort characters by class. & Mobile Device Simulator / Linux / Windows & - List of available races \& classes \newline - Set of characters with different names, races, and classes & Race Filter, Class Filter, Race \& Class Filter, Name Sorting, Race Sorting, Class Sorting \\ \hline
Character Deletion: \newline 1. Delete a created character from the list. & Mobile Device Simulator / Linux / Windows & - & Character Deletion \\ \hline
\end{tabular}
\end{table}

\end{document} 