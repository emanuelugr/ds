\documentclass{article}
%Para imagenes
\usepackage{graphicx}
\usepackage{float}
\usepackage{enumitem} % Paquete para personalizar listas
\usepackage[margin=2cm]{geometry} % Ajusta todos los márgenes a 2 centímetros
\usepackage{indentfirst} % Este paquete fuerza la sangría del párrafo en la primera línea
\setlength{\parindent}{1cm} % Ajusta la sangría de los párrafos a 1 cm


\begin{document}
    \begin{titlepage}
        \centering
        {\bfseries\LARGE Universidad de Granada\par}
        \vspace{1cm}
        {\scshape\Large Facultad de Ingeniería Informática \par}
        \vspace{2cm}
        {\scshape\Huge Practica 3: Pruebas software \par}
        \begin{figure}[h]
                \centering
                \includegraphics[width=0.6\textwidth]{logo_UGR.jpg}
                \label{fig:portada}
            \end{figure}
        {\itshape\Large DS: Grupo 1.7\par}
        \vfill
            {\Large  Emanuel Giraldo Herrera\par}
            {\Large  Thomas Lang \par}
            {\Large  Timur Sorokin \par}
            {\Large  Alejando Iborra Morán \par}
        \vfill
        {\Large (2023-2024) \par}
    \end{titlepage}

\tableofcontents

\newpage
\section{Planificación}

\subsection{"Creador de personajes"}

\begin{itemize}
	\item \textbf{Test 1}: CFG de ClaseBuilder\\
\textbf{Propósito}: Verificar que la función loadCFG() de ClaseBuilder funciona correctamente.

	\item \textbf{Test 2}: CFG de PersonajeBuilder
\\ \textbf{Propósito}: Confirmar que la función loadCFG() de PersonajeBuilder funciona correctamente.


	\item \textbf{Test 3}: Creación correcta del personaje\\
\textbf{Propósito}: Asegurar que la creación de un personaje utilizando PersonajeBuilder se realiza correctamente.


	\item \textbf{Test 4}: Exportación Correcta del personaje
\\\textbf{Propósito}: Verificar que la exportación del personaje se realiza correctamente.


	\item \textbf{Test 5}: Funcionamiento del director
\\ \textbf{Propósito}: Confirmar que el director puede crear un personaje correctamente.


	\item \textbf{Test 6}: Funcionamiento de la fachada
\\ \textbf{Propósito}: Verificar que la fachada funciona como se espera.

\end{itemize}

\subsection{"Gestor de personajes"}
	\begin{itemize}


	\item \textbf{Test 1}: Añadir Personaje
\\ \textbf{Propósito}: Confirmar que se pueden agregar personajes a la lista del gestor.


	\item \textbf{Test 2}: Eliminar Personaje
\\ \textbf{Propósito}: Verificar que se pueden eliminar personajes de la lista del gestor.


	\item \textbf{Test 3}: Operaciones sobre la lista
\\ \textbf{Propósito}: Confirmar que las operaciones sobre la lista de personajes funcionan correctamente.

\newpage
\section{Análisis}
\subsection{CFG de ClaseBuilder}

\textbf{Elemento a Probar}: 

La función loadCFG() de la clase ClaseBuilder.

\textbf{Condiciones para la Prueba}:

        Se requiere que exista un archivo de configuración válido para la clase en cuestión.
        La función loadCFG() debe cargar correctamente los valores del archivo de configuración.
   
\textbf{ Datos Necesarios para la Prueba:}
        Un archivo de configuración válido que contenga los atributos de la clase con sus respectivos valores.
        Valores esperados para cada atributo de la clase según el archivo de configuración.

\subsection{CFG de PersonajeBuilder}

\textbf{Elemento a Probar}:

 La función loadCFG() de la clase PersonajeBuilder.

\textbf{Condiciones para la Prueba}:

        Se requiere que exista un archivo de configuración válido para el personaje en cuestión.
        La función loadCFG() debe cargar correctamente los valores del archivo de configuración.
        
    \textbf{Datos Necesarios para la Prueba:}
    
        Un archivo de configuración válido que contenga los atributos del personaje con sus respectivos valores.
        Valores esperados para cada atributo del personaje según el archivo de configuración.

\subsection{Creación correcta del personaje}

   \textbf{ Elemento a Probar}: 
   
   El proceso de creación de un personaje utilizando PersonajeBuilder.
   
    \textbf{Condiciones para la Prueba:}
        Se deben cargar correctamente los valores de los archivos de configuración de la clase y el personaje.
        Los atributos del personaje creado deben coincidir con los valores esperados según las configuraciones de clase y personaje.
        
    \textbf{Datos Necesarios para la Prueba:}
        Archivos de configuración válidos para la clase y el personaje.
        Valores esperados para los atributos del personaje basados en las configuraciones de clase y personaje.

\subsection{Exportación Correcta del personaje}


    \textbf{Elemento a Probar}:
    
     El proceso de exportación de un personaje.
    
   \textbf{ Condiciones para la Prueba:}
   
        El personaje debe haber sido creado correctamente.
        La función de exportación debe generar un archivo con la estructura correcta y los datos del personaje.
        
    \textbf{Datos Necesarios para la Prueba:}
    
        Un personaje creado y válido.
        Valores esperados para los atributos del personaje en el archivo de exportación.

\subsection{Funcionamiento del director}

   \textbf{ Elemento a Probar:} 
   
   El proceso de creación de un personaje a través del director.
   
   \textbf{ Condiciones para la Prueba:}
        El director debe ser capaz de crear un personaje utilizando un constructor específico.
        
    \textbf{Datos Necesarios para la Prueba:}
    
        Un constructor válido para el tipo de personaje deseado.
        Valores esperados para los atributos del personaje creado.

\subsection{Funcionamiento de la fachada}

   \textbf{ Elemento a Probar:} 
   El funcionamiento de la fachada en una operación específica.
   
    \textbf{Condiciones para la Prueba:}
    
        La fachada debe ser capaz de realizar la operación deseada correctamente.
        
   \textbf{ Datos Necesarios para la Prueba:}
   
        Datos de entrada necesarios para la operación.
        Valores esperados para el resultado de la operación.

\end{itemize}

\section{Diseño}
\subsection{Creador de personajes}
\subsubsection{Grupo: Archivos de configuración}
\textbf{TLDR}:
Estos casos de prueba aseguran que los archivos de configuración necesarios para construir personajes estén presentes en el sistema y puedan ser accedidos correctamente por el programa.

\begin{itemize}
	\item \textbf{Casos de prueba}
	\begin{itemize}
		\item Configuración de clase: verificar que los archivos de configuración de clase existen en la ruta especificada
		\item Configuración de raza: verificar que los archivos de configuración de raza existen en la ruta especificada
	\end{itemize}
	\item \textbf{Entorno de prueba}\\
	No se requieren herramientas ni infraestructuras adicionales.

	\item \textbf{Datos de prueba}
	\begin{itemize}
		\item 	Se requiere ClaseBuilder debidamente inicializado.
			\item Se requiere PersonajeBuidler debidamente inicializado.
			\item Se requieren rutas donde se ubican los archivos de configuración.
	\end{itemize}

	
	\item \textbf{Condición base}\\
	Los archivos lib/cfg/* existen y son accesibles. 
\end{itemize}


\subsubsection{Grupo: ClaseBuilder CFG}
\textbf{TLDR:} Verifica que la función \texttt{loadCFG()} de ClaseBuilder carga correctamente los atributos de clase.

\begin{itemize}
	\item \textbf{Casos de prueba:}
	\begin{itemize}
		\item Lectura de atributos de clase: verificar que la función \texttt{loadCFG()} de ClaseBuilder funciona correctamente y carga los atributos de clase esperados.
	\end{itemize}
	
	\item \textbf{Entorno de prueba:}
	\begin{itemize}
		\item No se requieren herramientas ni infraestructuras adicionales.
	\end{itemize}
	
	\item \textbf{Datos de prueba:}
	\begin{itemize}
		\item Se requiere ClaseBuilder debidamente inicializado.
	\end{itemize}
	
	\item \textbf{Condición base:}
	\begin{itemize}
		\item ClaseBuilder está inicializado correctamente y ha cargado los valores desde el archivo de configuración correctamente.
	\end{itemize}
\end{itemize}


\subsubsection{Grupo: PersonajeBuilder CFG}
\textbf{TLDR:} Verifica que la función \texttt{loadCFG()} de PersonajeBuilder carga correctamente los atributos de personaje.

\begin{itemize}
	\item \textbf{Casos de prueba:}
	\begin{itemize}
		\item Lectura de atributos de personaje: verificar que la función \texttt{loadCFG()} de PersonajeBuilder funciona correctamente y carga los atributos de personaje esperados.
	\end{itemize}
	
	\item \textbf{Entorno de prueba:}
	\begin{itemize}
		\item No se requieren herramientas ni infraestructuras adicionales.
	\end{itemize}
	
	\item \textbf{Datos de prueba:}
	\begin{itemize}
		\item Se requiere PersonajeBuilder debidamente inicializado.
	\end{itemize}
	
	\item \textbf{Condición base:}
	\begin{itemize}
		\item PersonajeBuilder está inicializado correctamente  y cha cargado los valores desde el archivo de configuración correctamente.
	\end{itemize}
\end{itemize}


\subsubsection{Grupo: Creación de Personaje}
\textbf{TLDR:} Verifica que la creación de un personaje utilizando los constructores de ClaseBuilder y PersonajeBuilder produce un personaje con los atributos esperados.

\begin{itemize}
	\item \textbf{Casos de prueba:}
	\begin{itemize}
		\item Verificar que los atributos del personaje son los esperados después de su creación.
	\end{itemize}
	
	\item \textbf{Entorno de prueba:}
	\begin{itemize}
		\item No se requieren herramientas ni infraestructuras adicionales.
	\end{itemize}
	
	\item \textbf{Datos de prueba:}
	\begin{itemize}
		\item Se requieren ClaseBuilder y PersonajeBuilder debidamente inicializados.
	\end{itemize}
	
\item \textbf{Condición base:}
\begin{itemize}
	\item ClaseBuilder y PersonajeBuilder están inicializados correctamente y la construcción de los atributos en el builder da el mismo resultado que \texttt{loadCFG()}.
\end{itemize}
\end{itemize}


\subsubsection{Grupo: Exportación del estado}
\textbf{TLDR:} Verifica que la exportación del estado de un personaje se realiza correctamente y que se genera el archivo esperado.

\begin{itemize}
	\item \textbf{Casos de prueba:}
	\begin{itemize}
		\item Verificar que el archivo de exportación del estado del personaje se genera correctamente en la ubicación especificada.
	\end{itemize}
	
	\item \textbf{Entorno de prueba:}
	\begin{itemize}
		\item No se requieren herramientas ni infraestructuras adicionales.
	\end{itemize}
	
	\item \textbf{Datos de prueba:}
	\begin{itemize}
		\item Se requiere un personaje correctamente construido y una ubicación de archivo de exportación válida.
	\end{itemize}
	
	\item \textbf{Condición base:}
	\begin{itemize}
		\item El personaje está correctamente construido y la ubicación de archivo de exportación es válida. Finalmente que el archivo efectivamente se ha creado.
	\end{itemize}
\end{itemize}



\subsubsection{Grupo: Funcionamiento Director}
\textbf{TLDR:} Verifica que el director pueda crear un personaje correctamente y que el personaje creado tenga el nombre esperado.

\begin{itemize}
	\item \textbf{Casos de prueba:}
	\begin{itemize}
		\item Verificar que el director pueda crear un personaje con el nombre especificado.
		\item Verificar que el nombre del personaje creado coincida con el nombre especificado.
	\end{itemize}
	
	\item \textbf{Entorno de prueba:}
	\begin{itemize}
		\item No se requieren herramientas ni infraestructuras adicionales.
	\end{itemize}
	
	\item \textbf{Datos de prueba:}
	\begin{itemize}
		\item Se requiere un director correctamente inicializado y un nombre de personaje válido.
	\end{itemize}
	
	\item \textbf{Condición base:}
	\begin{itemize}
		\item El director está correctamente inicializado y puede crear un personaje con el nombre especificado.
	\end{itemize}
\end{itemize}



\subsubsection{Grupo: Funcionamiento de la Fachada}
\textbf{TLDR:} Verifica que la fachada pueda crear un personaje correctamente con los constructores proporcionados y que el personaje creado tenga el nombre esperado.

\begin{itemize}
	\item \textbf{Casos de prueba:}
	\begin{itemize}
		\item Verificar que la fachada pueda crear un personaje utilizando los constructores de ClaseBuilder y PersonajeBuilder.
		\item Verificar que el nombre del personaje creado coincida con el nombre especificado.
	\end{itemize}
	
	\item \textbf{Entorno de prueba:}
	\begin{itemize}
		\item No se requieren herramientas ni infraestructuras adicionales.
	\end{itemize}
	
	\item \textbf{Datos de prueba:}
	\begin{itemize}
		\item Se requiere una fachada correctamente inicializada, un constructor de ClaseBuilder y PersonajeBuilder válidos, y un nombre de personaje válido.
	\end{itemize}
	
	\item \textbf{Condición base:}
	\begin{itemize}
		\item La fachada está correctamente inicializada y puede crear un personaje utilizando los constructores proporcionados.
	\end{itemize}
\end{itemize}

\subsection{Gestor personajes}
Se encarga de verificar el funcionamiento adecuado del gestor de personajes en el sistema. Este gestor gestiona una la lista de personajes, permitiendo añadir, eliminar y realizar operaciones como ordenar y filtrar por diferentes criterios, como nombre, clase, raza o atributo.

\subsubsection{Añadir Personaje}
\textbf{TLDR:} Este caso de prueba verifica que el gestor de personajes pueda añadir correctamente un personaje a la lista de personajes.

\begin{itemize}
	\item \textbf{Casos de prueba:}
	\begin{itemize}
		\item Se añade un personaje a la lista del gestor de personajes.
	\end{itemize}
	
	arduino
	
	\item \textbf{Entorno de prueba:}
	\begin{itemize}
		\item Se necesita un gestor de personajes inicializado y un personaje válido.
	\end{itemize}
	
	\item \textbf{Datos de prueba:}
	\begin{itemize}
		\item Gestor de personajes correctamente inicializado.
		\item Personaje válido.
	\end{itemize}
	
	\item \textbf{Condición base:}
	\begin{itemize}
		\item El gestor de personajes está correctamente inicializado y la lista de personajes está vacía.
	\end{itemize}
	
\end{itemize}


\end{document} 