\documentclass{article}
%Para imagenes
\usepackage{graphicx}
\usepackage{float}
\usepackage{enumitem} % Paquete para personalizar listas
\usepackage[margin=2cm]{geometry} % Ajusta todos los márgenes a 2 centímetros
\usepackage{indentfirst} % Este paquete fuerza la sangría del párrafo en la primera línea
\setlength{\parindent}{1cm} % Ajusta la sangría de los párrafos a 1 cm


\begin{document}
    \begin{titlepage}
        \centering
        {\bfseries\LARGE Universidad de Granada\par}
        \vspace{1cm}
        {\scshape\Large Facultad de Ingeniería Informática \par}
        \vspace{2cm}
        {\scshape\Huge Practica 3: Pruebas software \par}
        \begin{figure}[h]
                \centering
                \includegraphics[width=0.6\textwidth]{logo_UGR.jpg}
                \label{fig:portada}
            \end{figure}
        {\itshape\Large DS: Grupo 1.7\par}
        \vfill
            {\Large  Emanuel Giraldo Herrera\par}
            {\Large  Thomas Lang \par}
            {\Large  Timur Sorokin \par}
            {\Large  Alejando Iborra Morán \par}
        \vfill
        {\Large (2023-2024) \par}
    \end{titlepage}

\tableofcontents

\newpage
\section{Planificación}

\subsection{"Creador de personajes"}

\begin{itemize}
	\item \textbf{Test 1}: CFG de ClaseBuilder\\
\textbf{Propósito}: Verificar que la función loadCFG() de ClaseBuilder funciona correctamente.

	\item \textbf{Test 2}: CFG de PersonajeBuilder
\\ \textbf{Propósito}: Confirmar que la función loadCFG() de PersonajeBuilder funciona correctamente.


	\item \textbf{Test 3}: Creación correcta del personaje\\
\textbf{Propósito}: Asegurar que la creación de un personaje utilizando PersonajeBuilder se realiza correctamente.


	\item \textbf{Test 4}: Exportación Correcta del personaje
\\\textbf{Propósito}: Verificar que la exportación del personaje se realiza correctamente.


	\item \textbf{Test 5}: Funcionamiento del director
\\ \textbf{Propósito}: Confirmar que el director puede crear un personaje correctamente.


	\item \textbf{Test 6}: Funcionamiento de la fachada
\\ \textbf{Propósito}: Verificar que la fachada funciona como se espera.

\end{itemize}

\subsection{"Gestor de personajes"}
	\begin{itemize}


	\item \textbf{Test 1}: Añadir Personaje
\\ \textbf{Propósito}: Confirmar que se pueden agregar personajes a la lista del gestor.


	\item \textbf{Test 2}: Eliminar Personaje
\\ \textbf{Propósito}: Verificar que se pueden eliminar personajes de la lista del gestor.


	\item \textbf{Test 3}: Operaciones sobre la lista
\\ \textbf{Propósito}: Confirmar que las operaciones sobre la lista de personajes funcionan correctamente.

\newpage
\section{Análisis}
\subsection{CFG de ClaseBuilder}

\textbf{Elemento a Probar}: 

La función loadCFG() de la clase ClaseBuilder.

\textbf{Condiciones para la Prueba}:

        Se requiere que exista un archivo de configuración válido para la clase en cuestión.
        La función loadCFG() debe cargar correctamente los valores del archivo de configuración.
   
\textbf{ Datos Necesarios para la Prueba:}
        Un archivo de configuración válido que contenga los atributos de la clase con sus respectivos valores.
        Valores esperados para cada atributo de la clase según el archivo de configuración.

\subsection{CFG de PersonajeBuilder}

\textbf{Elemento a Probar}:

 La función loadCFG() de la clase PersonajeBuilder.

\textbf{Condiciones para la Prueba}:

        Se requiere que exista un archivo de configuración válido para el personaje en cuestión.
        La función loadCFG() debe cargar correctamente los valores del archivo de configuración.
        
    \textbf{Datos Necesarios para la Prueba:}
    
        Un archivo de configuración válido que contenga los atributos del personaje con sus respectivos valores.
        Valores esperados para cada atributo del personaje según el archivo de configuración.

\subsection{Creación correcta del personaje}

   \textbf{ Elemento a Probar}: 
   
   El proceso de creación de un personaje utilizando PersonajeBuilder.
   
    \textbf{Condiciones para la Prueba:}
        Se deben cargar correctamente los valores de los archivos de configuración de la clase y el personaje.
        Los atributos del personaje creado deben coincidir con los valores esperados según las configuraciones de clase y personaje.
        
    \textbf{Datos Necesarios para la Prueba:}
        Archivos de configuración válidos para la clase y el personaje.
        Valores esperados para los atributos del personaje basados en las configuraciones de clase y personaje.

\subsection{Exportación Correcta del personaje}


    \textbf{Elemento a Probar}:
    
     El proceso de exportación de un personaje.
    
   \textbf{ Condiciones para la Prueba:}
   
        El personaje debe haber sido creado correctamente.
        La función de exportación debe generar un archivo con la estructura correcta y los datos del personaje.
        
    \textbf{Datos Necesarios para la Prueba:}
    
        Un personaje creado y válido.
        Valores esperados para los atributos del personaje en el archivo de exportación.

\subsection{Funcionamiento del director}

   \textbf{ Elemento a Probar:} 
   
   El proceso de creación de un personaje a través del director.
   
   \textbf{ Condiciones para la Prueba:}
        El director debe ser capaz de crear un personaje utilizando un constructor específico.
        
    \textbf{Datos Necesarios para la Prueba:}
    
        Un constructor válido para el tipo de personaje deseado.
        Valores esperados para los atributos del personaje creado.

\subsection{Funcionamiento de la fachada}

   \textbf{ Elemento a Probar:} 
   El funcionamiento de la fachada en una operación específica.
   
    \textbf{Condiciones para la Prueba:}
    
        La fachada debe ser capaz de realizar la operación deseada correctamente.
        
   \textbf{ Datos Necesarios para la Prueba:}
   
        Datos de entrada necesarios para la operación.
        Valores esperados para el resultado de la operación.

\end{itemize}

\end{document} 